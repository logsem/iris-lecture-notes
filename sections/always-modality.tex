\section{The Persistently Modality}
\label{sec:introducing-persistently}

\newcommand{\persduprule}[1][]
{\rulegen[#1]{persistently-dup}
  {}{\persistently P \provesIff \persistently P \ast P}}
\newcommand{\persintrorule}[1][]
{\rulegen[#1]{persistently-intro}
  {\persistently P \proves Q}{\persistently P \proves \persistently Q}}
\newcommand{\persseprule}[1][]
{\rulegenhref[#1]{persistently-sep}{Persistently-sep}
  {S \proves \persistently{\prop} \land \propB}
  {S \proves \persistently{\prop} \ast\propB}}
\newcommand{\perssepderivedrule}[1][]
{\rulegenhref[#1]{persistenly-sep-derived}{Persistently-sep-derived}
  {S \proves \persistently{(\prop \land \propB)}}
  {S \proves \persistently{(\prop \ast\propB)}}}
\newcommand{\persmonorule}[1][]
{\rulegen[#1]{persistently-mono}
{\prop \proves \propB}{\persistently{\prop} \proves \persistently{\propB}}}
\newcommand{\perselimrule}[1][]
{\rulegen[#1]{persistently-E}
{}{\persistently\prop \proves \prop}}
\newcommand{\persidemprule}[1][]
{\rulegen[#1]{persistently-idemp}
{}{\persistently{\prop} \proves \persistently\persistently\prop}}
\newcommand{\htpersrule}[1][]
{\rulegenb[#1]{Ht-persistently}
{\persistently\propB \land S \proves \hoare{\prop}{\expr}{\Ret\val.\propC}}
    { S \proves \hoare{\prop \land \persistently{\propB}}{\expr}{\Ret\val.\propC}}}
\newcommand{\htreclob}[1][]
{\rulegen[#1]{Ht-rec-lob}
{ \Gamma, f : \Val \mid Q \land \All y . \All v. \hoare{\later P}{f v}{ \Phi } \proves \All y . \All v. \hoare{P}{e[v/x]}{\Phi}}
    { \Gamma \mid Q \proves \All y . \All v. \hoare{\later P}{(\Rec{f} x = e) v}{\Phi}}}
\newcommand{\persandrule}[1][]
  {\rulegenhref[#1]{persistently-$\wedge$}{persistently-and}{}
   {\persistently(\prop \land \propB) \provesIff \persistently\prop \land \persistently\propB}}
\newcommand{\persorrule}[1][]
{\rulegenhref[#1]{persistently-$\vee$}{persistently-or}{}
{\persistently(\prop \lor \propB) \provesIff \persistently\prop \lor \persistently\propB}}
\newcommand{\perslaterrule}[1][]
{\rulegenhref[#1]{persistently-$\later$}{persistently-later}{}
{\persistently\later \prop \provesIff \later\persistently\prop}}
\newcommand{\persforallrule}[1][]
{\rulegenhref[#1]{persistently-$\forall$}{persistently-forall}{}
{\All x. \persistently{\prop} \provesIff \persistently{\All x. \prop}}}
\newcommand{\persexistsrule}[1][]
{\rulegenhref[#1]{persistently-$\exists$}{persistently-exists}{}
{\persistently{\Exists x. \prop} \provesIff \Exists x. \persistently{\prop}}}
\newcommand{\perstruerule}[1][]
{\rulegen[#1]{persistently-True}{}{\TRUE \provesIff \persistently{\TRUE}}}
\newcommand{\perstermeqrule}[1][]
{\rulegen[#1]{persistently-Eq}{}{t =_\tau t' \provesIff \persistently (t =_\tau t')}}
\newcommand{\pershtrule}[1][]
{\rulegen[#1]{persistently-Ht}{}{\hoare{P}{e}{\Phi} \provesIff \persistently \hoare{P}{e}{\Phi}}}
The rules \ruleref{Ht-Ht} and \ruleref{Ht-Eq} and their generalizations involving $\later$ used in Section~\ref{sec:introducing-later} are somewhat ad-hoc.
For instance, they do not allow us to move disjunctions of persistent propositions in and out of preconditions.
This could be addressed by having a separate category of propositions which could be moved in and out of preconditions, and this category would be closed under all connectives of higher-order logic.
However it would still not address the issue of how to ensure that a proposition is persistent \emph{inside the logic}.
This latter issue is important when using the logic for more advanced examples.
For instance it is crucial for defining Hoare triples from the notion of weakest precondition which we consider in Section~\ref{sec:first-steps-towards-base-logic}, and for other more advanced uses of the logic.
In this section we present a more uniform way of treating persistent propositions, which also addresses the second issue.
We introduce a new modality $\persistently$ (pronounced ``persistently'').

The intuitive reading of $\persistently P$ is that it contains only those resources of $P$ which can be split into a duplicable resource $s$ in $P$ and some other resource $r$, \ie{} those which have a duplicable part in $P$.
Thus $\persistently P$ is like $P$ except that it does not assert any exclusive ownership over resources.
An example of a duplicable resource is the empty heap.
If resources are heaps then that is the only duplicable resource.
As a concrete example of where this is not the case (this example is also related to the notion of invariants we introduce later) we can imagine a slight refinement of heaps as a notion of resource.
Suppose that each heap is split into two parts $(h_r, h_w)$, where $h_r$ is read-only, and $h_w$ is an ordinary heap which supports reading and writing.
Composition of $(h_r, h_w)$ and $(h_r', h_w')$ is only defined when the components $h_r$ and $h_r'$ are the same, and moreover the composition of $h_w$ and $h_w'$ is defined as composition of ordinary heaps.
Then the duplicable part of the element $(h_r, h_w)$ is the element $(h_r, \emptyset)$, where $\emptyset$ is the empty heap.

Duplicable resources are important because we can always create a copy of them to give away to other threads.
One way to state this precisely is the following rule
\begin{mathpar}
  \persduprule[-inline]
\end{mathpar}
This rule is derivable from the axioms for the persistently modality, which
are shown in Figure~\ref{fig:laws-for-persistently}.
We discuss these below.

We call a proposition $P$ \emph{persistent} if it satisfies $P \proves \persistently P$.
Persistent propositions are important because they are duplicable, see
Exercise~\ref{ex:always:derived}.

An example of a persistent proposition is the equality relation.
A \emph{non-example} of a persistent proposition is the points-to predicate $x \pointsto v$.
Indeed, any heap in $x \pointsto v$ contains at least one location, namely $x$.
Hence the empty heap is not in $x \pointsto v$.
By the same reasoning we have $\persistently (x \pointsto v) \provesIff \FALSE$.

Given the above intuitive reading of $\persistently P$,
the first three axioms for $\persistently$ in Figure~\ref{fig:laws-for-persistently},
\ruleref{persistently-mono}, \ruleref{persistently-E} and \ruleref{persistently-idemp},
are immediate.
These three rules together allow us to prove the 
following ``introduction'' rule for the persistently modality.
\begin{mathpar}
  \persintrorule
\end{mathpar}

\begin{exercise}
  Prove the above introduction rule.
\end{exercise}


The rules in the right column of Figure~\ref{fig:laws-for-persistently}
state that $\TRUE$ is persistent.  If resources are heaps then $\TRUE$
is the set of all heaps.  Hence $\TRUE$ in particular contains the
empty heap and thus $\persistently \TRUE = \TRUE$.  The rest of the rules
state that $\persistently$ commutes with the propositional connectives.  We
do not explain why that is the case since that would require us to
describe a model of the logic.

Next comes the rule \ruleref{Persistently-sep} and the derived rule \ruleref{Persistently-sep-derived}.
They govern the interaction between separating conjunction, conjunction, and the persistently modality.
\begin{mathpar}
  \persseprule[-inline]
\end{mathpar}
The intuitive reason for why this rule 
holds is that any resource $r$ in $\persistently P$ and $Q$ can be split
into a duplicable resource $s$ in $P$ (and since it is duplicable
also in $\persistently P$) and $r$.
Hence $r$ is in $\persistently{\prop} \ast\propB$.

There is an important derived rule which states that under the persistently
modality conjunction and separating conjunction coincide.  The rule is
\begin{mathpar}
  \perssepderivedrule
\end{mathpar}
which is equivalent to the entailment
\begin{align*}
  \persistently{(\prop \land \propB)} \proves \persistently{(\prop \ast\propB)}.
\end{align*}
To derive this entailment we first show the following:
\begin{align}
  \label{eq:persistently-duplicable}
  \infer{P \proves \persistently P}{P \proves P \ast P}.
\end{align}
Indeed, using $P \proves \persistently P$ we have
\begin{align*}
  P \proves \persistently P \proves \persistently P \land \persistently P \proves \persistently P \ast \persistently P \proves P \ast P
\end{align*}
using the rule \ruleref{Persistently-sep} in the third step, and $\persistently P \proves P$ and monotonicity of separating conjunction in the last.

Next, by the fact that $\persistently$ commutes with conjunction and 
\eqref{eq:persistently-duplicable}, we have
\begin{align*}
  \persistently P \land \persistently Q \proves \persistently (P \land Q) \proves \persistently (P \land Q) \ast \persistently (P \land Q)
\end{align*}
Now, using the fact that $P \land Q \proves P$ and
$P \land Q \proves Q$ and monotonicity of $\persistently$ and $\ast$, we have
\begin{align*}
  \persistently (P \land Q) \ast \persistently (P \land Q) \proves \persistently P \ast \persistently Q.
\end{align*}
Hence we have proved
\begin{align}
  \label{eq:persistently-land-sep-coincide}
  \persistently P \land \persistently Q \proves \persistently P \ast \persistently Q.
\end{align}
With this we can finally derive \ruleref{Persistently-sep-derived} as follows
\begin{align*}
  \persistently(P \land Q) \proves \persistently\persistently(P \land Q) \proves \persistently(\persistently P \land \persistently Q) \proves \persistently(\persistently P \ast \persistently Q) \proves \persistently(P \ast Q)
\end{align*}
where in the last step we again use $\persistently P \proves P$ and monotonicity of separating conjunction.

\begin{figure}[htbp]
  \begin{mathpar}
    \persduprule
    \and
    \persmonorule
    \and
    \perselimrule
    \and
    \persidemprule
    \and
    \persseprule
  \end{mathpar}
  The persistently modality additionally has the following structural rules.
  \begin{mathpar}
    \persandrule
    \and
    \persorrule
    \and
    \perslaterrule
    \\
    \persforallrule
    \and
    \persexistsrule
  \end{mathpar}
  Finally we have three kinds of primitive persistent propositions.
  \begin{mathpar}
    \perstruerule
    \and
    \perstermeqrule
    \and
    \pershtrule
  \end{mathpar}
  We have the following rule \ruleref{Ht-persistently} which, combined with the rules above, generalizes \ruleref{Ht-Ht}, \ruleref{Ht-Eq} and \ruleref{Ht-False}.
  \begin{mathparpagebreakable}
    \htpersrule
  \end{mathparpagebreakable}
  \caption{Laws for the persistently modality.}
  \label{fig:laws-for-persistently}
\end{figure}

\begin{exercise}
  \label{ex:always:derived}
  Using similar reasoning show the following derived rules.
  \begin{enumerate}
  \item $\persistently\persistently P \proves \persistently P$
  \item $\persistently (P \Ra Q) \proves \persistently P \Ra \persistently Q$
  \item $P \Ra Q \proves P \wand Q$
  \item $\persistently (P \wand Q) \proves \persistently (P \implies Q)$
  \item $\persistently (P \wand Q) \proves \persistently P \wand \persistently Q$
  \item $\persistently P \implies Q \proves \persistently P \wand Q$
  \item if $P \proves \persistently Q$ then $P \proves \persistently Q \ast P$.
  \item $(P \wand (Q \ast \persistently R)) \ast P \proves (P \wand (Q \ast \persistently R)) \ast P \ast \persistently R$
  \end{enumerate}
  The last two items are often useful.
  They state that we can get persistent propositions without consuming resources.
\end{exercise}


\begin{exercise}
  Derive $\persistently(x \pointsto v) \proves \FALSE$ using the rules in Figure~\ref{fig:laws-for-persistently} and the basic axioms of the points-to predicate.
\end{exercise}


\begin{exercise}
  \label{exercise:derived-rule-recursive-functions}
  Show the following derived rule for recursive function calls from the rule \ruleref{Ht-Rec} above.
  \begin{mathpar}
    \htreclob
  \end{mathpar}
  Hint: Start with L\"ob induction.
  Use the rule \ruleref{Ht-persistently} to move the L\"ob induction hypothesis into the precondition.
  Then use the recursion rule \ruleref{Ht-Rec}, and again \ruleref{Ht-persistently}.
  You are almost there.
\end{exercise}

%%% Local Variables:
%%% mode: latex
%%% TeX-master: "../main.tex"
%%% End:
