\section{Case Study: foldr}
\label{sec:case-study:foldr}
In this section we will use the logic introduced thus far to give a very general higher order specification of the \langkw{foldr} function on linked lists.
We then use this specification to verify two clients using the \langkw{foldr} function, the \langkw{sumList} function, which computes the sum of all the elements in the linked list, and the \langkw{filter} function, which creates a new list with only those elements of the original list which satisfy the given predicate.

The implementation of the \langkw{foldr} function is as follows.
\begin{displaymath}
  \Rec{\langkw{foldr}} f, a, l  =
  \MatchML{l}with{\Inj{1} x_1}=>{a}|{\Inj{2} x_2}=>{
    \begin{array}[t]{l} 
      \Let h = \Proj{1} \deref x_2 in\\
      \Let t = \Proj{2} \deref x_2 in\\
      f\, (h, \langkw{foldr} (f, a, t)) \end{array}}end{}
\end{displaymath}
The first argument $f$ is a function taking a pair as an argument, the second argument $a$ is the result of $\langkw{foldr}$ on the empty list, and $l$ is the linked list.

The specification of the $\langkw{foldr}$ function is as follows.
\begin{align*}
\All P, I. \All f \in \Val . \All xs. \All l. \hoareV[t]
{
  \begin{array}{l}
    \isList {l} {xs} \ast \operatorname{all} P\, xs \ast  I \, [] \, a \ast{}\\
    \left( \All x \in \Val. \All a' \in \Val. \All ys. \hoare{P x \ast I \, ys \, a'}{f \, (x, \, a')}{r. I (x::ys) r} \right)
  \end{array}
  }
{\langkw{foldr}(f, a, l)}
{r.  \isList {l} {xs} \ast I\, xs\, r}
\end{align*}
where the predicate $\operatorname{all} P\, xs$ states that $P$ holds for all elements of the list.
It is defined by induction on the list as
\begin{align*}
 \operatorname{all} P\, [] &\equiv \TRUE \\
 \operatorname{all} P\, (x::xs) &\equiv P \, x \ast \operatorname{all} P\, xs
\end{align*}

We want the specification of $\langkw{foldr}$ to capture to following:
\begin{itemize}
\item The third argument $l$ needs to be a linked list implementing the mathematical sequence $xs$.
  This is captured by $\isList {l} {xs}$ in the precondition.
\item We do not want $\langkw{foldr}$ to change the list.
  This is captured by having $\isList {l} {xs}$ in the postcondition.
\item Some clients may not want to operate on general lists but only on subsets of those (\eg{}~only on lists of odd natural numbers or lists of booleans).
  This is captured by letting the client choose a predicate $P$ and then having $\operatorname{all} P\, xs$ in the precondition force the specification to only hold for list $xs$, where $P \ x$ holds for all $x$ in $xs$.
\item We want to be able to relate the return value $r$ to the list $xs$ we have folded.
  This is done by an invariant $I\, xs\, r$.
  In the case were $\langkw{foldr}$ is applied to the empty list the return value is simply the $a$ provided, hence $I$ needs to be chosen such that $I \, [] \, a$ holds.
  This is expressed by the assertion $I\, []\, a$ in the precondition.

  The last assertion in the precondition is the specification of the function $f$ which is used to combine the elements of the list, we remark more on its specification below.
\end{itemize}
The specification of $\langkw{foldr}$ is higher-order in the sense that it involves nested Hoare triples (here in the precondition).
The reason being that $\langkw{foldr}$ takes a function $f$ as argument, hence we can't specify $\langkw{foldr}$ without having some knowledge or specification for the function $f$.
Different clients may instantiate $\langkw{foldr}$ with some very different functions, hence it can be hard to give a specification for $f$ that is reasonable and general enough to support all these choices.
In particular knowing when one has found a good and provable specification can be difficult in itself.

The specification for $f$ that we have chosen is
\begin{align*}
\All x. \All a'. \All ys. \hoare{P x \ast I \ ys \ a'}{f \ (x, \ a')}{r. I\ (x::ys)\   r}
\end{align*}
We explain the specification by how it is used in the proof of $\langkw{foldr}$, which is detailed below.
In the proof of $\langkw{foldr}$ we are only going to use the specification for $f$ when $l$ represents the sequence $(x::xs)$, where the sequence $xs$ is represented by some implementation $t$.
In this case we are going to instantiate the specification of $f$ with $\langkw{foldr}(f, a, t)$ as $a'$, $xs$ as $ys$ and $x$ as $x$.

In this sense the specification for $f$ simply says that if $x$ satisfies $P$ and the result $a'$ of folding $f$ over the subsequence $xs$ satisfies the invariant, then applying $f$ to $(x, a')$ (which is exactly what $\langkw{foldr}$ does on the sequence $(x::xs)$) gives a result, such that $I$ relates it to the sequence $(x::xs)$.

\begin{remark}
  The only place, where our concrete implementation $l$ of the list (as a linked list) is used is in $\isList {l} {xs}$ where it is linked to a mathematical sequence $xs$.
  The predicates $P, I$ and $\operatorname{all}$ are all defined on mathematical sequences instead.
  In this sense we may say that the mathematical sequences are abstractions or models of our concrete lists and $P, I$ and $\operatorname{all}$ operate on these models, hence the distinction between implementation details and mathematical properties becomes clear.
\end{remark}

\subsubsection*{Proof of the specification}
Let $P, I, f, xs$ and $l$ be given. As Hoare triples are persistent and persistence is preserved by quantification we may move 
\begin{align*}
\All x. \All a'. \All ys. \hoare{P x \ast I \, ys \, a'}{f \, (x, \, a')}{r. I\, (x::ys)}
 \end{align*} 
 into the context i.e. we may assume it. We thus have to prove
\begin{align*}
\All x. \All a'. \All ys. \hoare{P \, x \ast I \, ys \, a'}{f \, (x, \, a')}{r. I\, (x::ys)}
\proves \hoareV
{ \isList {l} {xs} \ast \operatorname{all} P \, xs \ast  I \, [] \, a}
{\langkw{foldr}(f, a, l)}
{r.  \isList {l} {xs} \ast I \, xs\, r}
\end{align*}

We proceed by \ruleref{Ht-Rec} i.e. we have to prove the specification for the body of $\langkw{foldr}$ in which we have replaced any occurrence of $\langkw{foldr}$ with the function $g$ for which
\begin{align*}
\All xs. \All l. \hoare{ \isList {l} {xs} \ast \operatorname{all} P \, xs \ast  I \, [] \, a}
 {g(f, a, l)}
{r.  \isList {l} {xs} \ast I \, xs\, r}
\end{align*}
is assumed.

By the definition of the $\operatorname{isList}$ predicate we have that the list $l$ points to is either empty $[]$ or has a head $x$ followed by another list $xs'$.  

In the first case we need to show
\begin{align*}
\hoare{ \isList {l} {[]} \ast \operatorname{all} P \, [] \ast  I \, [] \, a}
{\langkw{match}\ l\ \langkw{with} ...}
{r.  \isList {l} {[]} \ast I \, []\, r} 
\end{align*}
and in the second
\begin{align*}
\hoare{ \isList {l} {(x::xs')} \ast \operatorname{all} P \, (x::xs') \ast  I \, [] \, a}
{\langkw{match}\ l\ \langkw{with} ...}
{r.  \isList {l} {xs} \ast I \, xs\, r}
\end{align*}
In both cases we proceed by the derived match rule from Exercise~\ref{exercise:derived-match-rule}.

In the first case we have $\isList {l} {[]} \equiv l = \langkw{inj}_1()$ hence the first case is taken (follows by \ruleref{Ht-frame}, \ruleref{Ht-Pre-Eq} and \ruleref{Ht-ret}), thus we have to prove
\begin{align*}
\hoare{ \isList {l} {[]} \ast  I \, [] \, a}{a}{r.  \isList {l} {[]} \ast I \, []\, r} 
\end{align*}
After framing $ \isList {l} {[]} $ we are left with proving
\begin{align*}
\hoare{ I \, [] \, a}{a}{r. I\, [] \, r} 
\end{align*}
As $I \, [] \, r $ follows from $r = a \ast I \, [] \, a$ it suffices by \ruleref{Ht-csq} to show
\begin{align*}
\hoare{ I \, [] \, a}{a}{r.r=a \ast I \, [] \, a }
\end{align*}
which follows by \ruleref{Ht-frame} and \ruleref{Ht-ret}.

In the second case we have $\isList {l} {(x :: xs')} \equiv \Exists hd, l'.l = \langkw{inj}_2 hd \ast hd \pointsto (x, l') \ast \isList {l'} {xs'}$.
By exercise \ref{exercise:isList-second-case} we get that the second case is taken, thus we need to prove
\begin{align*}
  \hoareV{ 
  	\isList {l'} {xs'} \ast \operatorname{all} P \, (x::xs') \ast I \, [] \, a \ast l = \Inj2 hd \ast hd \pointsto (x,l') }
  {\begin{array}{l}
     \Let h= \Proj{1} \deref x_2 in\\
     \Let t = \Proj{2} \deref x_2 in\\
    	f\, (h, (g(f, a, t)))
   \end{array}}
  {r.  \isList {l} {(x::xs')} \ast I \, (x::xs') \, r}
\end{align*}
By simple applications of \ruleref{Ht-let-det-temp}, \ruleref{Ht-frame}, \ruleref{Ht-bind}, \ruleref{Ht-load-temp}, \ruleref{Ht-Proj} and \ruleref{Ht-ret} we need to show:
\begin{align*}
  \hoareV{ \isList {l'} {xs'} \ast \operatorname{all} P \, (x::xs) \ast I \, [] \, a \ast l = \Inj2 hd \ast hd \pointsto (x,l')  }
    	{f (x, g(f, a, l'))}
  {r. \isList {l} {(x::xs')} \ast I \, (x::xs') \, r}
\end{align*}
By \ruleref{Ht-bind-det} we have to show:
\begin{enumerate}
\item $f\, (x , -)$ is an evaluation context.
\item the specification
  \begin{align*}
\hoareV{ \isList {l'} {xs'} \ast \operatorname{all} P \, (x::xs') \ast I \, [] \, a \ast l = \Inj2 hd \ast hd \pointsto (x,l')  }
    	{g(f, a, l')}
  {r. \isList {l'} {xs'} \ast I \, xs' \, r \ast P\, x \ast l = \Inj2 hd \ast hd \pointsto (x,l') }
\end{align*}
\item the specification \begin{align*}
  \hoareV{ \isList {l'} {xs'} \ast I \, xs' \, v \ast P\, x \ast l = \Inj2 hd \ast hd \pointsto (x,l') }
    	{f (x, v)}
  {r. \isList {l} {(x::xs')} \ast I \, (x::xs') \, r}
\end{align*}
\end{enumerate}

\begin{enumerate}
\item As $f$ is a value by assumption it follows that $f \, E$ is an evaluation context, when $E$ is.
  Now as $x$ is a value we have that $(x, -)$ is an evaluation context hence it follows that $f (x, - )$ is an evaluation context.%
  \footnote{If we hadn't let $f$ take the arguments as a tuple then we would have been stuck here as $f (x, -)$ is not an evaluation context.
  To get past this, we would need to change the specification, such that $f \, x$ would return a function $g$, that when applied to $a'$ would satisfy the current specification for $f$ \ie{} we would have to specify $f$ using a nested Hoare triple and the specification for \langkw{foldr} would then involve a nested triple inside a nested triple.}

\item Follows by unfolding the definition of $\operatorname{all} P (x::xs') = P \, x \ast \operatorname{all} P \, xs'$, framing $P \, x \ast l = \Inj2 hd \ast hd \pointsto (x,l')$ and then using our assumption on $g$.

\item As $ \isList {l'} {xs'} \ast l = \Inj2 hd \ast hd \pointsto (x,l') \implies \isList {l} {(x::xs')} $ then it suffices by \ruleref{Ht-csq} to show
  \begin{align*}
    \hoareV{ \isList {l} {(x::xs')} \ast P\, x \ast I \, xs' \, v  }
    {f x \, v}
    {r. \isList {l} {(x::xs')} \ast I \, (x::xs') \, r}
  \end{align*}
  which follows by framing $\isList {l} {(x::xs')}$ and using our assumption
  on $f$.
\end{enumerate}

\subsubsection*{Client: sumList}
The following client is a function that computes the sum of a list of natural numbers by making a right-fold on +, 0 and the list. The code below is  a slightly longer as it has to take into account that $f$ takes the arguments as a pair. 
\begin{displaymath}
    \langkw{sumList} \, l =
    \begin{array}[t]{l}
    \Let f = (\lambda p, \Let x = \Proj{1} p in \Let y = \Proj{2} p in x + y)\\
   		 in \langkw{foldr}(f, 0, l)
    \end{array}
  \end{displaymath}

The specification of the $\langkw{sumList}$ function is as follows.
\begin{align*}
\All l. \All xs.
  \hoare{ \isList {l} {xs} \ast \operatorname{all} \operatorname{isNat}\, xs}{\langkw{sumList} \, l}{r.  \isList {l} {xs} \ast r = \Sigma_{x \in xs} x}
\end{align*}
where 
\begin{align*}
\operatorname{isNat} \, x &\equiv \begin{cases} \TRUE & \text{if } x \in \mathbb{N} \\
																			\FALSE & \text{otherwise } \end{cases}
\end{align*}
is a predicate stating that the argument is a natural number.

The specification of the $\langkw{sumList}$ function states that given a list of natural numbers implemented by $l$, the result of $\langkw{sumList}\,l$ is the sum of all the natural numbers in the list.
The $\isList {l} {xs}$ in the postcondition again ensures that $\langkw{sumList}$ does not change the list.

\paragraph*{Proof of the \langkw{sumList} specification}
Let $l$ and  $xs$ be given. By \ruleref{Ht-let-det-temp} it suffices to show
\begin{align*}
  \hoareV{ \isList {l} {xs} \ast \operatorname{all} \operatorname{isNat} xs}
  {\langkw{foldr}\left((\lambda p, \Let x = \Proj{1} p in \Let y = \Proj{2} p in x + y), 0, l\right)}
  {r.  \isList {l} {xs} \ast r = \Sigma_{x \in xs} x}
\end{align*}
Using the specification for $\langkw{foldr}$, with $\operatorname{isNat}$ as $P$, $a' = \Sigma_{y \in ys } y$ as $I \, ys \, a'$, $0$ as $a$,
\begin{align*}
  (\lambda p, \Let x = \Proj{1} p in \Let y = \Proj{2} p in x + y)
\end{align*}
as $f$, $l$ as $l$, and $xs$ as $xs$ we get 
\begin{align*}
\hoareV{ \begin{array}{l}
\left( \All x, a. \All ys.  \hoare{\operatorname{isNat}\, x \ast a = \Sigma_{y \in ys} y}{(\lambda p, \Let x = \Proj{1} p in \Let y = \Proj{2} p in x + y) (x, \, a)}{r. r = \Sigma_{y\in (x::ys)}y} \right) \\
\ast \isList {l} {xs} \ast \operatorname{all} \operatorname{isNat}\, xs \ast 0 = \Sigma_{x \in []} x\end{array}}
{\langkw{foldr}\left((\lambda p, \Let x = \Proj{1} p in \Let y = \Proj{2} p in x + y), a, l\right)}
{r.  \isList {l} {xs} \ast r = \Sigma_{x \in xs} x}
\end{align*}
which is almost what we want. The difference being the precondition. By \ruleref{Ht-csq} it suffices to show
\begin{align*}
\isList {l} {xs} \ast \operatorname{all} \operatorname{isNat} xs \implies 
\begin{array}{l}
\left( \All x, a. \All ys.  \hoareV{\operatorname{isNat}\, x \ast a = \Sigma_{y \in ys} y}{(\lambda p, \Let x = \Proj{1} p in \Let y = \Proj{2} p in x + y) (x, \, a)}{r. r = \Sigma_{y\in (x::ys)}y} \right) \\
\ast \isList {l} {xs} \ast \operatorname{all} \operatorname{isNat} xs \ast 0 = \Sigma_{x \in []} x\end{array}
\end{align*}
\ie{}~it is suffices to prove  
\begin{enumerate}
\item $\All x, a. \All ys. \hoare{\operatorname{isNat}\ x \ast a = \Sigma_{y \in ys} y}{(\lambda p, \Let x = \Proj{1} p in \Let y = \Proj{2} p in x + y) (x, a)}{r. r = \Sigma_{y\in (x::ys)}y}$
  \label{enum:item1}
\item $0 = \Sigma_{x \in []} x$
\end{enumerate}
without assuming anything. 

The second item is immediate, and we leave the first as an exercise.
\begin{exercise}
Prove the specification \eqref{enum:item1} above.
\end{exercise}



\subsubsection*{Client: filter}
The following client implements a filter of some boolean predicate $p$ on a list $l$, \ie{} it creates a \emph{new list} whose elements are precisely those elements of $l$ that satisfy $p$.
\begin{displaymath}
    {\langkw{filter} (p, l) =}
    \begin{array}[t]{l}
      \Let f = (\lambda y, \begin{array}[t]{l}
                             \Let x = \Proj{1}  y in\\
                             \Let xs = \Proj{2} y  in\\
                             \If {p\, x} \\
                             then \Inj{2} (\Ref (x, xs) ) \\
                             \Else  xs
                             )
                           \end{array}\\
      in \\
      \langkw{foldr}(f, \Inj{1}(), l)
    \end{array}
  \end{displaymath}
  
 \noindent \textbf{Specification}
\begin{align*}
\All P. \All l. \All xs. 
\hoareV[t]{ \left( \All x. \hoare{\TRUE}{p \, x}{v. \operatorname{isBool} \, v \ast v = P \, x}\right)
\ast \isList {l} {xs} }{\langkw{filter}(p, l)}{r.  \isList {l} {xs} \ast \operatorname{isList} \, r \left( \listFilter {P} {xs} \right)}
\end{align*}
where
\begin{align*}
\listFilter {P} {[]} &\equiv \, [] \\
\listFilter {P} {(x :: xs)} &\equiv \begin{cases} \left( x :: \left( \listFilter {P} {xs} \right) \right)
													  & \text{if } \, P \, x = \True  \\
\listFilter {P} {xs}  & \text{otherwise } \end{cases}
\end{align*}

The specification
\begin{align*}
  \All x. \hoare{\TRUE}{p \, x}{v. \operatorname{isBool} \, v \ast v = P \, x}
\end{align*}
in the precondition of the specification of $\langkw{filter}$ states that $p$ implements the boolean predicate $P$.
The specification of $\langkw{filter}$ states that given such an implementation $p$ and a list $l$, whose elements corresponds to the mathematical sequence $xs$, then \langkw{filter} returns (without changing the original list) a list $r$ whose elements are those that satisfy $P$.

\paragraph*{Proof of the specification of \langkw{filter}}
Let $P, l$ and $xs$ be given. By $\ruleref{Ht-let-det-temp}$ it suffices to show
\begin{align*}
\hoareV{ \left( \All x. \hoare{\TRUE}{p \, x}{v. \operatorname{isBool} \, v \ast v = P \, x}\right)
\ast \isList {l} {xs} }
{\langkw{foldr}\left(\left( \lambda y, \Let x = \Proj{1} y in \Let xs = \Proj{2} y in \If {p\, x} then \Inj{2} (\Ref (x, xs) ) \Else  xs\right), \Inj{1}(), l\right)}{r.  \isList {l} {xs} \ast \operatorname{isList} \, r \left( \listFilter {P} {xs} \right)}
 \end{align*}
  By applying the specification for \langkw{foldr} with $\TRUE$ as $P \, x $, $\isList {a} {\left( \listFilter {P} {xs} \right)} $ as $I \, xs \, a $,
  \begin{align*}
    \left( \lambda y, \Let x = \Proj{1} y in \Let xs = \Proj{2} y in \If {p\, x} then \Inj{2} (\Ref (x, xs) ) \Else  xs \right)
  \end{align*}
  as $f$, $xs$ as $xs$ and $l$ as $l$ we get that
 \begin{align*}
\hoareV{ \begin{array}{l}
\left(  \All x'. \All a'. \All ys. \hoare {\TRUE \ast \isList {a'} {\left( \listFilter {P} {ys} \right)}}{p' (x', a')}{v. \isList {v} {\left( \listFilter {P} {(x'::ys)} \right)} }
\right) \\
\ast \isList {l} {xs} \ast \operatorname{all} (\lambda x . \TRUE) \, xs \ast \isList {(\Inj{1}())} {(\listFilter {P} {[]})} \end{array}}
{\langkw{foldr}\left(p', \Inj{1}(), l\right)}{r.  \isList {l} {xs} \ast \isList {r} {\left( \listFilter {P} {xs} \right)}}
 \end{align*}
 where $p'$ is a shorthand notation for
 \begin{align*}
   \lambda y, \Let x = \Proj{1} y in \Let xs = \Proj{2} y in \If {p\, x} then \Inj{2} (\Ref (x, xs) ) \Else  xs.
 \end{align*}
 
 By \ruleref{Ht-csq} we are done if we can show that
 \begin{align*}
 & \left( \All x. \hoare{\TRUE}{p \, x}{v. \operatorname{isBool} \, v \ast v = P \, x}\right)
\ast \isList {l} {xs}  \\
& \implies \\
& \begin{array}{l}
\left(  \All x'. \All a'. \All ys. \hoare {\TRUE \ast \isList {a'} {\left( \listFilter {P} {ys} \right)}}{p' \, (x', \, a')}{v.\isList {v} {\left( \listFilter {P} {(x'::ys)} \right)}}
\right) \\
\ast \isList {l} {xs} \ast \operatorname{all} (\lambda x . \TRUE) \, xs \ast \isList {(\Inj{1}())} {(\listFilter {P} {[]})} \end{array}
 \end{align*}
Since $\isList {l} {xs}$ clearly implies $\isList {l} {xs}$ we only need to show.
\begin{enumerate}
\item $\All x. \hoare{\TRUE}{p \, x}{v. \operatorname{isBool} \, v \ast v = P \, x} \\
\implies \All x'. \All a'. \All ys. \hoare {\TRUE \ast \isList {a'} {\left( \listFilter {P} {ys} \right)}}{p' \, (x', \, a')}{v.\isList {v} {\left( \listFilter {P} {(x'::ys)} \right)}}$
\item $\TRUE \implies \operatorname{all} (\lambda x . \TRUE) \, xs$
\item $\TRUE \implies \isList {(\Inj{1}())} {(\listFilter {P} {[]})}$
\end{enumerate}

\begin{exercise}
  Prove the preceding three implications.\\
  Hint: use induction for the second item.
\end{exercise}

%%% Local Variables:
%%% mode: latex
%%% TeX-master: "../main.tex"
%%% End:
