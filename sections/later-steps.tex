
\section{The later modality and steps of computation}
\label{sec:later-steps}

When explaining the later modality and weakest preconditions we explained informally that
the later modality is tied to physical steps of computation (\ie{} steps in the small-step operational semantics of the programming language).
Formally, this can be seen in the actual definition of weakest preconditions, which we do not present in these notes but which can be seen in \cite{iris-ground-up}.
That being said, it is very instructive to look at this relationship within the logic, \ie{} using just what we have seen so far.
To shed some light on this relationship, in this section we look at an example of a program that is only safe, \ie{} does not get stuck, for a few steps of computation, and see what we can show in \Iris{} about such a program.
First, however, we take a closer look at how we can formally state in \Iris{} that a statement holds for a certain number of steps.

In the formal language of \Iris{} we formalize a proposition holding for a number of steps as follows:
\begin{definition} Given a proposition $\prop$ we say that $\prop$ holds for $k$ steps if and only if the following holds:
\begin{align*}
  {\later}^{k} \FALSE \proves \prop \tag{$\prop$ holds for $k$ steps}
\end{align*}
\end{definition}
This makes intuitive sense: if $\prop$ holds under the assumption that $\FALSE$ holds after $k$ steps, then $\prop$ must hold for $k$ steps.
In addition, we have that for a proposition $\prop$,
\begin{align}
\proves \prop \text{ holds if and only if for \emph{any} } k \text{ we have } {\later}^{k} \FALSE \proves \prop \text{ holds} \label{eq:holds-if-for-all-steps}
\end{align}
This again makes intuitive sense: $\prop$ holds if and only if it holds for arbitrarily many steps.
The \emph{only if} direction of the above statement \eqref{eq:holds-if-for-all-steps} is trivial.
The \emph{if} direction, on the other hand, is not.
It follows, rather easily (left as an exercise), from the following:
\begin{align}
  \proves \exists k.\; {\later}^{k} \FALSE \label{eq:expose-steps}
\end{align}
The statement \eqref{eq:expose-steps} above, which might be a bit surprising at first sight, is in fact logically equivalent to \ruleref{Loeb} induction.
Before showing this however, let us ponder the intuitive meaning of \eqref{eq:expose-steps}.
It says that there is exists a number $k$ such that $\FALSE$ holds after $k$ many steps.
Recall that \Iris{} is a step-indexed logic where the intuitive meaning of $\proves \prop$ is that for any $j$, $\prop$ holds for $j$ steps.
With this intuition in mind, \eqref{eq:expose-steps} essentially says that for any $j$ the following holds for $j$ steps: there exists a number $k$ such that after $k$ steps $\FALSE$ holds.
Since we are considering the truth of the statement ``there exists a number $k$ \dots'' only up to $j$ steps, we can simply take $k$ to be any number greater than $j$.
Then, by definition, we don't care about what happens after $k$ steps because it is already beyond the $j$ many steps that we care about.
The crucial point here is the interpretation of the existential quantifier in our step-indexed logic.
It allows us to pick different $k$'s for the different number of steps being considered.
A deep and thorough consideration of the intuitive meaning behind the proof of the following theorem should reveal the intuitive reasoning we just discussed for why \eqref{eq:expose-steps} holds.

\begin{remark}
  Note that the statement \eqref{eq:expose-steps} holding, \ie{} $\proves \exists k.\; {\later}^{k} \FALSE$, does not imply that there is in fact some number $n$ for which $\proves {\later}^{n} \FALSE$ holds.
  The key point here is that the existential quantifier can be instantiated to be a different natural number depending on the current step-index.
  The proposition $\proves {\later}^{n} \FALSE$ cannot hold (\ie{} cannot hold for all step-indices) for any fixed $n$ because for any fixed $n$ it would not hold for step-indices greater than $n$.
\end{remark}

\begin{theorem}
  Statement \eqref{eq:expose-steps} is equivalent to the principle of \ruleref{Loeb} induction.
\end{theorem}

\begin{proof}
{\bfseries [\eqref{eq:expose-steps} \emph{is implied by} \ruleref{Loeb}]}
We need to show $\proves \exists k.\; {\later}^{k} \FALSE$.
By \ruleref{Loeb} induction it suffices to show that $\later \exists k.\; {\later}^{k} \FALSE \proves \exists k.\; {\later}^{k} \FALSE$.
Now, we can use the \ruleref{later-exists} rule.
Hence, we have to show that $\exists k.\; {\later}^{k+1} \FALSE \proves \exists k.\; {\later}^{k} \FALSE$.
We then eliminate the existential quantifier as the natural number $l$ which leaves us to show ${\later}^{l+1} \FALSE \proves \exists k.\; {\later}^{k} \FALSE$.
We simply take $k$ to be $l + 1$, or any number larger than that, to finish the proof.\\

\noindent
{\bfseries [\eqref{eq:expose-steps} \emph{implies} \ruleref{Loeb}]}
We need to show that $\propB \proves \prop$ given that $\propB \land \later \prop \proves \prop$.
Now, by \eqref{eq:expose-steps} we can add $\exists k.\; {\later}^{k} \FALSE$ to our hypotheses, since we are assuming it holds.
Thus it suffices to show that $\propB \land \exists k.\; {\later}^{k} \FALSE \proves \prop$.
We proceed by eliminating the existential quantifier as the natural number $l$ which leaves us to show $\propB \land {\later}^{l} \FALSE \proves \prop$.
We proceed by induction on $l$.
The base case is trivial: we have to show that $\propB \land \FALSE \proves \prop$.
For the inductive case, we can assume (induction hypothesis) that $\propB \land {\later}^{l} \FALSE \proves \prop$ and have to show $\propB \land {\later}^{l+1} \FALSE \proves \prop$.
Now, by our starting hypothesis, (the antecedent of \ruleref{Loeb}) we have that $\propB \land \later \prop \proves \prop$.
Hence, it suffices to show that $\propB \land {\later}^{l+1} \FALSE \proves \propB \land \later \prop$ to conclude the proof.
Obviously, $\propB \land {\later}^{l+1} \FALSE \proves \propB$ holds.
Therefore, we only need to show $\propB \land {\later}^{l+1} \FALSE \proves \later \prop$.
We proceed by weakening $\propB$ by putting it under a later modality.
That is, we need to show $\later \propB \land {\later}^{l+1} \FALSE \proves \later \prop$.
And since later commutes with conjunction (\ruleref{Later-conj}) we simply need to show $\later (\propB \land {\later}^{l} \FALSE) \proves \later \prop$.
Finally, the \ruleref{Later-Mono} rule can be applied which leaves us to show $\propB \land {\later}^{l} \FALSE \proves \prop$.
This is precisely our induction hypothesis (the induction we performed on $l$) which concludes the proof.
\end{proof}

To summarize, what we have seen so far captures, in the formal language of the logic, what we intuitively think when we think about step-indexing in \Iris{}.
The proposition ${\later}^{k} \FALSE$ holds if we are proving things only up to $k$ steps.
When we are proving a proposition $\prop$ in \Iris{} we are essentially showing that for any $k$ the proposition $\prop$ holds for $k$ steps \eqref{eq:holds-if-for-all-steps}.
And that there is always a step-index $k$ such that we are proving things only up to $k$ steps \eqref{eq:expose-steps}.
Note how the latter intuition essentially tells us that the reason why \ruleref{Loeb} induction holds is that we can perform induction on the step-index; this, \ie{} induction on the step-index, is exactly how one proves the principle of \ruleref{Loeb} induction sound in the model of \Iris{} \cite{iris-ground-up}.

Now, let us consider the following program which is \emph{not} safe.
It crashes, \ie{} gets stuck, but only after two steps of computation.
\[ \langkw{let}~ \mathit{twoSteps} = (\lambda \var. \If \var then 2 \Else 3 + \var)~\False \]
Since \Iris{} ties physical steps of computation (in the sense of small-step operational semantics) to logical steps, \ie{} the later modality, we should be able to show the following:
\begin{align*}
  \proves \hoare{{\later}^{2} \FALSE}{\mathit{twoSteps}}{\var.\; \FALSE}
\end{align*}
And we can.
After applying the rule \ruleref{Ht-beta} corresponding to the function application step we need to show
\begin{align*}
  \proves \hoare{\later \FALSE}{\If \False then 2 \Else 3 + \False}{\var.\; \FALSE}
\end{align*}
Now we can use the rule \ruleref{Ht-If} for conditionals which leaves us to show
\begin{align*}
  \proves \hoare{\FALSE \ast \False = \True}{3 + \False}{\var.\; \FALSE}
\end{align*}
and
\begin{align*}
  \proves \hoare{\FALSE \ast \False = \False}{3 + \False}{\var.\; \FALSE}
\end{align*}
Both of which hold vacuously.
Note how in this proof the postcondition is irrelevant as the program never terminates within the two steps of computation that we are considering it.

%%% Local Variables:
%%% mode: latex
%%% TeX-master: "../main.tex"
%%% End:
